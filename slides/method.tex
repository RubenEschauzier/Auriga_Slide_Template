\begin{frame}{Desiderata of a retrieval order metric}
    \begin{itemize}
        \item Measures only retrieval performance
        \item Should indicate how far from optimal the current approach is
    \end{itemize}
\end{frame}

\begin{frame}{The $ R^{3} $ Metric}
    \begin{itemize}
        \item Introducing: Relevant Retrieval Ratio ($ R^{3} $)
        \item Measures the ratio of relevant dereference events compared to the optimal dereference order
    \end{itemize}
\end{frame}

\begin{frame}{The $ R^{3} $ Metric}
    \begin{columns}[T] % align columns
        \begin{column}{.48\textwidth}

       \begin{figure}
            \centering
            \includegraphics[height = .7\textheight]{figures/traversal_start}
        \end{figure}

        \end{column}%
        \hfill%
        \begin{column}{.48\textwidth}
            \bigskip
            \begin{itemize}
                \item The queried subweb and query-relevant documents determine algorithmic performance
            \end{itemize}
        \end{column}%
    \end{columns}
\end{frame}

\begin{frame}{The $ R^{3} $ Metric - BFS vs DFS}
    \begin{columns}[T] % align columns
        \begin{column}{.48\textwidth}

       \begin{figure}
            \centering
            \includegraphics[height = .7\textheight]{figures/traversal_breadth_first}
        \end{figure}

        \end{column}%
        \hfill%
        \begin{column}{.48\textwidth}
            \begin{figure}
                \centering
                \includegraphics[height = .7\textheight]{figures/traversal_depth_first}
            \end{figure}    
        \end{column}%
    \end{columns}
\end{frame}


\begin{frame}{The $ R^{3} $ Metric - Optimal}
    \begin{columns}[T] % align columns
        \begin{column}{.48\textwidth}

       \begin{figure}
            \centering
            \includegraphics[height = .7\textheight]{figures/traversal_optimal}
        \end{figure}

        \end{column}%
        \hfill%
        \begin{column}{.48\textwidth}
            \bigskip
            \begin{itemize}
                \item The Traversal graph is a directed graph
                \item Equivalent to a Steiner tree problem
                \item Reuse existing optimizers for this problem
            \end{itemize}
        \end{column}%
    \end{columns}
\end{frame}

\begin{frame}{The $ R^{3} $ Metric}
    \begin{columns}[T] % align columns
        \begin{column}{.48\textwidth}
            \bigskip
            \bigskip
            \bigskip
            \begin{aligned}
                R^{3} = \dfrac{|T_{O}|}{|T_{E}|}, \quad \quad |T_{O}|, \: |T_{E}| > 0
            \end{aligned}               
        \end{column}%
        \hfill%
        \begin{column}{.48\textwidth}
            \begin{itemize}
                \item $|T_{O}|$ the length of the optimal path
                \item $|T_{E}|$ the length of the taken path
                \item Higher is better
            \end{itemize}
        \end{column}%
    \end{columns}
\end{frame}

\begin{frame}{The $ R^{3} $ Metric}
    \begin{itemize}
        \item BFS traverses $ 10 $ links, DFS $ 7 $, and optimal $ 3 $  
        \item BFS: $R^{3} = 3 / 10 = 0.30 $
        \item DFS: $R^{3} = 3 / 7 = 0.43 $
    \end{itemize}
\end{frame}