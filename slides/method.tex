\begin{frame}{Modeling Query Usage Patterns}
    \begin{itemize}
        \item Extend decentralized benchmark SolidBench 
        \item Identify and simulate client-specific query usage patterns
    \end{itemize}
    \note{
        \begin{itemize}
            \item SolidBench is a decentralized social media benchmark.
            \item Our modeling of query usage patterns will include a literature review on sub-community formation
            in social networks.
            \item And include analysis of real-life query logs to identify user types.
            \item Use this theoretical framework to extend SolidBench.
        \end{itemize}
    }
\end{frame}

\begin{frame}{Modeling Query Usage Patterns}
    \begin{figure}
        \centering
        \includegraphics[width = \textwidth]{figures/identify-query-patterns}
    \end{figure}
\end{frame}



\begin{frame}{Caching for LTQP}
    \begin{figure}
        \centering
        \includegraphics[width = \textwidth]{figures/caching-levels}
    \end{figure}
\end{frame}

\begin{frame}{Possible Cache Content}
    \begin{itemize}
        \item Dataset summaries
        \item Characteristic sets 
        \item Approximate Membership Functions 
        \item Indexes
    \end{itemize}
\end{frame}
    
\begin{frame}{Learned Query Optimization}
    \begin{itemize}
        \item Trained online
        \item Sample Efficiency
        \item Learned Query Hints
    \end{itemize}
        \note{
            \begin{itemize}
                \item As we don't know what data the user will query beforehand (this is different depending on the query), we must train the model online 
                \item This requires either using adaptive query processing with a learned optimizer recording information in the background or fully online training
                \item Both cases require high sample efficiency to offset the training cost 
                \item Learning query hints instead of full query optimization is shown to be significantly more sample efficient for relational databases.
            \end{itemize}
        }
\end{frame}
    
    
\begin{frame}{Learned Query Optimization}
    \begin{figure}
        \centering
        \includegraphics[width = \textwidth]{figures/LearnedQueryHints}
    \end{figure}
\end{frame}