\begin{frame}{Where is query-relevant data?}
    \begin{columns}[T] % align columns
        \begin{column}{.48\textwidth}

       \begin{figure}
            \centering
            \includegraphics[height = .7\textheight]{figures/where_is_my_data}
        \end{figure}

        \end{column}%
        \hfill%
        \begin{column}{.48\textwidth}
            \bigskip
            \begin{itemize}
                \item Data is stored in silos
                \item Reduces interopability
                \item Stifles innovation
            \end{itemize}
        \end{column}%
    \end{columns}
\end{frame}

\begin{frame}{Data Retrieval Order}
    \begin{columns}[T] % align columns
        \begin{column}{.48\textwidth}

       \begin{figure}
            \centering
            \includegraphics[height = .7\textheight]{figures/how_to_get_there_slow}
        \end{figure}

        \end{column}%
        \hfill%
        \begin{column}{.48\textwidth}
            \bigskip
            \begin{itemize}
                \item We first retrieve data irrelevant to the query
                \item Only after many jumps can we process data that can answer our query
            \end{itemize}
        \end{column}%
    \end{columns}
\end{frame}


\begin{frame}{Data Retrieval Order}
    \begin{columns}[T] % align columns
        \begin{column}{.48\textwidth}

       \begin{figure}
            \centering
            \includegraphics[height = .7\textheight]{figures/how_to_get_there_fast}
        \end{figure}

        \end{column}%
        \hfill%
        \begin{column}{.48\textwidth}
            \bigskip
            \begin{itemize}
                \item We quickly dereference data relevant to our query
            \end{itemize}
        \end{column}%
    \end{columns}
\end{frame}


\begin{frame}{The Impact of Data Retrieval Order}
    \begin{itemize}
        \item Previous work (CITE HERE) failed to find an algorithm that outperformed the baseline
        \item Why?
    \end{itemize}
\end{frame}
